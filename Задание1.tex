\documentclass[a4paper, 12pt]{article}
\usepackage[16pt]{extsizes}
\usepackage{mathtext}
\usepackage[T1,T2A]{fontenc}
\usepackage[english,russian]{babel}
\usepackage[left=12.7mm, top=12.7mm, right=12.7mm, bottom=12.7mm, nohead, footskip=5mm]{geometry} % настройки полей документа

%для вставки рисунков
\usepackage{graphicx}
\graphicspath{ {./images/} }

\usepackage[fleqn]{amsmath}

\title{Домашняя работа №1}
\author{А-13а-19 Самсонова Мария}

\begin{document}

\maketitle

\section{Построить конечный автомат}
  
\begin{description}

  \item [L_1 =\{ \omega \in \{a,b,c\}^* ||\omega_{c}| = 1 \}]
  
  \item \includegraphics[scale=0.25]{1}
  
  \item [L_2 = \{ \omega \in \{a,b\}^* || \omega_{a}| \leq 2 |\omega_{b}| \geq 2 \}]
  \item Воспользуемся прямым произведением к языкам $L_{21}$ и $L_{22}$
  \item $L_{21}=\{ \omega \in \{a\}^* || \omega_{a}| \leq 2\}$
  \item $L_{22}=\{ \omega \in \{b\}^* || \omega_{b}| \geq 2\}$
  
  \item \includegraphics[scale=0.55]{2}
  
  \item [L_3 = \{ \omega \in \{a,b\}^* ||\omega_{a}| \neq |\omega_{b}| \}]
    \item Воспользуемся дополнением к языку $L_3$
    \item $\overline{L_3} = \{ \omega \in \{a,b\}^* ||\omega_{a}| = |\omega_{b}| \}$
    \item Докажем, что $\overline{L_3}$ нерегулярный язык, для этого воспользуемся леммой о накачке
    \begin{itemize}
      \item Зафиксируем n
      \item Возьмем слово $a^nb^n \in \overline{L_3}$
      \item Разобьем его на $x,y,z$ такие, что $|xy| \leq n$ и $|y| \geq 1$
      \begin{itemize}
        \item $x=a^{n-m}$
        \item $y=a^m$
        \item $z=b^n$
      \end{itemize}
      \item Тогда при накачке $y$ полученное слово $\notin \overline{L_3}$ 
    \end{itemize}
    \item Из нерегулярности дополнения к языку следует нерегулярность языка
  
  \item [L_4 = \{ \omega \in \{a,b\}^* |\omega\omega = \omega\omega\omega \}]
  \item
  
  \item \includegraphics[scale=0.25]{4}
    
\end{description}    

\end{document}




