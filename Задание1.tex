\documentclass[a4paper, 12pt]{article}
\usepackage[16pt]{extsizes}
\usepackage{mathtext}
\usepackage[T1,T2A]{fontenc}
\usepackage[english,russian]{babel}
\usepackage[left=12.7mm, top=12.7mm, right=12.7mm, bottom=12.7mm, nohead, footskip=5mm]{geometry} % настройки полей документа

%для вставки рисунков
\usepackage{graphicx}
\graphicspath{ {./images/} }

\usepackage[fleqn]{amsmath}

\title{Домашняя работа №1}
\author{А-13а-19 Самсонова Мария}

\begin{document}

\maketitle

%1 задание
\section{Построить конечный автомат, распознающий язык}
  
\begin{description}

  \item [L_1 =\{ \omega \in \{a,b,c\}^* ||\omega_{c}| = 1 \}]
  
  \item \includegraphics[scale=0.25]{1_1}
  
  \item [L_2 = \{ \omega \in \{a,b\}^* || \omega_{a}| \leq 2 |\omega_{b}| \geq 2 \}]
  \item Воспользуемся прямым произведением к языкам $L_{21}$ и $L_{22}$
  \item $L_{21}=\{ \omega \in \{a\}^* || \omega_{a}| \leq 2\}$
  \item $L_{22}=\{ \omega \in \{b\}^* || \omega_{b}| \geq 2\}$
  
  \item \includegraphics[scale=0.55]{1_2}
  
  \item [L_3 = \{ \omega \in \{a,b\}^* ||\omega_{a}| \neq |\omega_{b}| \}]
    \item Воспользуемся дополнением к языку $L_3$
    \item $\overline{L_3} = \{ \omega \in \{a,b\}^* ||\omega_{a}| = |\omega_{b}| \}$
    \item Докажем, что $\overline{L_3}$ нерегулярный язык, для этого воспользуемся леммой о накачке
    \begin{itemize}
      \item Зафиксируем n
      \item Возьмем слово $a^nb^n \in \overline{L_3}$
      \item Разобьем его на $x,y,z$ такие, что $|xy| \leq n$ и $|y| \geq 1$
      \begin{itemize}
        \item $x=a^{n-m}$
        \item $y=a^m$
        \item $z=b^n$
      \end{itemize}
      \item Тогда при накачке $y$ полученное слово $\notin \overline{L_3}$ 
    \end{itemize}
    \item Из нерегулярности дополнения к языку следует нерегулярность языка
  
  \item [L_4 = \{ \omega \in \{a,b\}^* |\omega\omega = \omega\omega\omega \}]
  
  \item \includegraphics[scale=0.25]{1_4}
    
\end{description}    

\newpage
%2 задание
\section{Построить конечный автомат, используя прямое произведение}

\begin{description}

  \item [L_1 =\{ \omega \in \{a,b\}^* ||\omega_{a}| \geq 2 \land |\omega_{b}| \geq 2 \}]
  \item Воспользуемся прямым произведением к языкам $L_{11}$ и $L_{12}$
  \item $L_{11}=\{ \omega \in \{a\}^* || \omega_{a}| \geq 2\}$
  \item $L_{12}=\{ \omega \in \{b\}^* || \omega_{b}| \geq 2\}$
  \item Жирным выделена терминальная вершина
  \item \begin{tabular}{ | l | l | l | }
  \hline
    Узел & a & b \\ \hline
    q1q4 & q2q4 & q1q5 \\
    q1q5 & q2q5 & q1q6 \\
    q1q6 & q2q6 & q1q6 \\
    q2q4 & q3q4 & q2q5 \\
    q2q5 & q3q5 & q2q6 \\
    q2q6 & q3q6 & q2q6 \\
    q3q4 & q3q4 & q3q5 \\
    q3q5 & q3q5 & q3q6 \\
    \textbf {q3q6} & q3q6 & q3q6 \\
  \hline
  \end{tabular}
  \item \includegraphics[scale=0.65]{2_1}
  \newpage
  
  \item [L_2 = \{ \omega \in \{a,b\}^* || \omega | \leq 3 \land |\omega| нечетное \}]
  \item Воспользуемся прямым произведением к языкам $L_{21}$ и $L_{22}$
  \item $L_{21}=\{ \omega \in \{a,b\}^* || \omega | \leq 3\}$
  \item $L_{22}=\{ \omega \in \{a,b\}^* || \omega | нечетное\}$
  \item Жирным выделена терминальная вершина
  \item \begin{tabular}{ | l | l | l | }
  \hline
    Узел & a & b \\ \hline
    q1q5 & q2q6 & q2q6 \\
    q1q6 & q2q5 & q2q5 \\
    q2q5 & q3q6 & q3q6 \\
    q2q6 & q3q5 & q3q5 \\
    q3q5 & q4q6 & q4q6 \\
    q3q6 & q4q5 & q4q5 \\
    q4q5 & q4q6 & q4q6 \\
    \textbf {q4q6} & q4q5 & q4q5 \\
  \hline
  \end{tabular}
  \item Одна из ветвей графа является излишней, так как в нее невозможно попасть из начального узла q15
  \item \includegraphics[scale=0.65]{2_2}
  \includegraphics[scale=0.65]{2_2(f)}
  \newpage
  
  \item [L_3 = \{ \omega \in \{a,b\}^* ||\omega_{a}| четное \land |\omega_{b}| \vdots 3\}]
  \item Воспользуемся прямым произведением к языкам $L_{31}$ и $L_{32}$
  \item $L_{31}=\{ \omega \in \{a,b\}^* ||\omega_{a}| четное\}$
  \item $L_{32}=\{ \omega \in \{a,b\}^* ||\omega_{b}| \vdots 3\}$
  \item Жирным выделена терминальная вершина
  \item \begin{tabular}{ | l | l | l | }
  \hline
    Узел & a & b \\ \hline
    \textbf {q1q3} & q2q3 & q1q4 \\
    q1q4 & q2q4 & q2q5 \\
    q2q5 & q2q5 & q2q3 \\
    q2q3 & q1q3 & q2q4 \\
    q3q4 & q1q4 & q1q5 \\
    q3q5 & q1q5 & q1q3 \\
  \hline
  \end{tabular}
  \item \includegraphics[scale=0.45]{2_3}
  
  \item [L_4 = \overline{L_3}]
  \item Воспользуемся свойством обратного языка - заменим терминальные узлы на обычные, и наоборот - обычные на терминальные.
  \item \includegraphics[scale=0.45]{2_4}
  \newpage
  
  \item [L_5 = L_2 \setminus L_3]
  \item Разность двух языков можно представить как L_2 $\cap \overline{L_3}$
  \item \includegraphics[scale=0.4]{2_5}
  \item В графе содержатся лишние узлы, так как в них невозможно попасть из начального узла q16, удалим их
  \item \includegraphics[scale=0.4]{2_5(f)}
  \newpage
  
\end{description}

%3 задание
\section{Построить минимальный ДКА по регулярному выражению}
\begin{description}

  \item \includegraphics[scale=0.4]{3_1(l)}
  \includegraphics[scale=0.4]{3_1(d)}
  \includegraphics[scale=0.4]{3_1(f)}
  
  \item \includegraphics[scale=0.4]{3_2(l)}
  \includegraphics[scale=0.4]{3_2(d)}
  \includegraphics[scale=0.4]{3_2(f)}
  
  
\end{description}

\end{document}
