\documentclass[a4paper, 12pt]{article}
\usepackage[16pt]{extsizes}
\usepackage{mathtext}
\usepackage[T1,T2A]{fontenc}
\usepackage[english,russian]{babel}
\usepackage[left=12.7mm, top=12.7mm, right=12.7mm, bottom=12.7mm, nohead, footskip=5mm]{geometry} % настройки полей документа

%для вставки рисунков
\usepackage{graphicx}
\graphicspath{ {./images/} }

\usepackage[fleqn]{amsmath}

\title{Домашняя работа №1}
\author{А-13а-19 Самсонова Мария}

\begin{document}

\maketitle

%1 задание
\section{Построить конечный автомат, распознающий язык}
  
\begin{description}

  \item [L_1 =\{ \omega \in \{a,b,c\}^* ||\omega_{c}| = 1 \}]
  
  \item \includegraphics[scale=0.25]{1_1}
  
  \item [L_2 = \{ \omega \in \{a,b\}^* || \omega_{a}| \leq 2 |\omega_{b}| \geq 2 \}]
  \item Воспользуемся прямым произведением к языкам $L_{21}$ и $L_{22}$
  \item $L_{21}=\{ \omega \in \{a\}^* || \omega_{a}| \leq 2\}$
  \item $L_{22}=\{ \omega \in \{b\}^* || \omega_{b}| \geq 2\}$
  
  \item \includegraphics[scale=0.55]{1_2}
  
  \item [L_3 = \{ \omega \in \{a,b\}^* ||\omega_{a}| \neq |\omega_{b}| \}]
    \item Воспользуемся дополнением к языку $L_3$
    \item $\overline{L_3} = \{ \omega \in \{a,b\}^* ||\omega_{a}| = |\omega_{b}| \}$
    \item Докажем, что $\overline{L_3}$ нерегулярный язык, для этого воспользуемся леммой о накачке
    \begin{itemize}
      \item Зафиксируем n
      \item Возьмем слово $a^nb^n \in \overline{L_3}$
      \item Разобьем его на $x,y,z$ такие, что $|xy| \leq n$ и $|y| \geq 1$
      \begin{itemize}
        \item при 0 < m < n
        \item $x=a^{n-m}$ 
        \item $y=a^m$
        \item $z=b^n$
      \end{itemize}
      \item Тогда при накачке $y$ полученное слово $\notin \overline{L_3}$ 
    \end{itemize}
    \item Из нерегулярности дополнения к языку следует нерегулярность языка
  
  \item [L_4 = \{ \omega \in \{a,b\}^* |\omega\omega = \omega\omega\omega \}]
  
  \item \includegraphics[scale=0.25]{1_4}
    
\end{description}    

\newpage
%2 задание
\section{Построить конечный автомат, используя прямое произведение}

\begin{description}

  \item [L_1 =\{ \omega \in \{a,b\}^* ||\omega_{a}| \geq 2 \land |\omega_{b}| \geq 2 \}]
  \item Воспользуемся прямым произведением к языкам $L_{11}$ и $L_{12}$
  \item $L_{11}=\{ \omega \in \{a\}^* || \omega_{a}| \geq 2\}$
  \item $L_{12}=\{ \omega \in \{b\}^* || \omega_{b}| \geq 2\}$
  \item Жирным выделена терминальная вершина
  \item \begin{tabular}{ | l | l | l | }
  \hline
    Узел & a & b \\ \hline
    q1q4 & q2q4 & q1q5 \\
    q1q5 & q2q5 & q1q6 \\
    q1q6 & q2q6 & q1q6 \\
    q2q4 & q3q4 & q2q5 \\
    q2q5 & q3q5 & q2q6 \\
    q2q6 & q3q6 & q2q6 \\
    q3q4 & q3q4 & q3q5 \\
    q3q5 & q3q5 & q3q6 \\
    \textbf {q3q6} & q3q6 & q3q6 \\
  \hline
  \end{tabular}
  \item \includegraphics[scale=0.65]{2_1}
  \newpage
  
  \item [L_2 = \{ \omega \in \{a,b\}^* || \omega | \leq 3 \land |\omega| нечетное \}]
  \item Воспользуемся прямым произведением к языкам $L_{21}$ и $L_{22}$
  \item $L_{21}=\{ \omega \in \{a,b\}^* || \omega | \leq 3\}$
  \item $L_{22}=\{ \omega \in \{a,b\}^* || \omega | нечетное\}$
  \item Жирным выделена терминальная вершина
  \item \begin{tabular}{ | l | l | l | }
  \hline
    Узел & a & b \\ \hline
    q1q5 & q2q6 & q2q6 \\
    q1q6 & q2q5 & q2q5 \\
    q2q5 & q3q6 & q3q6 \\
    q2q6 & q3q5 & q3q5 \\
    q3q5 & q4q6 & q4q6 \\
    q3q6 & q4q5 & q4q5 \\
    q4q5 & q4q6 & q4q6 \\
    \textbf {q4q6} & q4q5 & q4q5 \\
  \hline
  \end{tabular}
  \item Одна из ветвей графа является излишней, так как в нее невозможно попасть из начального узла q15
  \item \includegraphics[scale=0.65]{2_2}
  \includegraphics[scale=0.65]{2_2(f)}
  \newpage
  
  \item [L_3 = \{ \omega \in \{a,b\}^* ||\omega_{a}| четное \land |\omega_{b}| \vdots 3\}]
  \item Воспользуемся прямым произведением к языкам $L_{31}$ и $L_{32}$
  \item $L_{31}=\{ \omega \in \{a,b\}^* ||\omega_{a}| четное\}$
  \item $L_{32}=\{ \omega \in \{a,b\}^* ||\omega_{b}| \vdots 3\}$
  \item Жирным выделена терминальная вершина
  \item \begin{tabular}{ | l | l | l | }
  \hline
    Узел & a & b \\ \hline
    \textbf {q1q3} & q2q3 & q1q4 \\
    q1q4 & q2q4 & q2q5 \\
    q2q5 & q2q5 & q2q3 \\
    q2q3 & q1q3 & q2q4 \\
    q3q4 & q1q4 & q1q5 \\
    q3q5 & q1q5 & q1q3 \\
  \hline
  \end{tabular}
  \item \includegraphics[scale=0.45]{2_3}
  
  \item [L_4 = \overline{L_3}]
  \item Воспользуемся свойством обратного языка - заменим терминальные узлы на обычные, и наоборот - обычные на терминальные.
  \item \includegraphics[scale=0.45]{2_4}
  \newpage
  
  \item [L_5 = L_2 \setminus L_3]
  \item Разность двух языков можно представить как L_2 $\cap \overline{L_3}$
  \item \includegraphics[scale=0.4]{2_5}
  \item В графе содержатся лишние узлы, так как в них невозможно попасть из начального узла q16, удалим их
  \item \includegraphics[scale=0.4]{2_5(f)}
  \newpage
  
\end{description}

%3 задание
\section{Построить минимальный ДКА по регулярному выражению}
\begin{description}
  \item [(ab + aba)∗a]
  \item Построим НКА с использованием $\lambda$-переходов, затем удалим их
  \item С помощью алгоритма Томсона построим ДКА
  \item Q: \{1\} \{4,7,12\} \{1,5,8\} \{1,4,7,9,12\}
  \item \begin{tabular}{ | l | l | l | }
  \hline
    Узел & a & b \\ \hline
    1 & 4,7,12 & \emptyset \\
    4,7,12 & \emptyset & 1,5,8 \\
    1,5,8 & 1,4,7,9,12 & \emptyset \\
    1,4,7,9,12 & 1,4,7,9,12 & 1,5,8 \\
  \hline
  \end{tabular} 
  \item Полученный ДКА минимальный
  \item \includegraphics[scale=0.4]{3_1(l)}
  \includegraphics[scale=0.4]{3_1(d)}
  \includegraphics[scale=0.4]{3_1(f)}
  \newpage
  
  \item [a(a(ab)∗b)∗(ab)∗]
  \item Построим НКА с использованием $\lambda$-переходов, затем удалим их
  \item \includegraphics[scale=0.4]{3_2(l)}
  \includegraphics[scale=0.4]{3_2(d)}
  
  \item С помощью алгоритма Томсона построим ДКА
  \item Q: \{1\} \{2\} \{4,12\} \{6\} \{9,13\} \{7\} \{9\}
  \item \begin{tabular}{ | l | l | l | }
  \hline
    Узел & a & b \\ \hline
    1 & 2 & \emptyset \\
    2 & 4,12 & \emptyset \\
    4,12 & 6 & 9,13 \\
    6 & \emptyset & 7 \\
    9,13 & 4,12 & \emptyset \\
    7 & 6 & 9 \\
    9 & 4,12 & \emptyset \\
  \hline
  \end{tabular}
  \item Минимизируем полученный ДКА
  \item \includegraphics[scale=0.4]{3_2(f)}
  \includegraphics[scale=0.4]{3_2(m)}
  \newpage
  
  \item [(a+(a+b)(a+b)b)∗]
  \item Построим НКА с использованием $\lambda$-переходов, затем удалим их
  \item \includegraphics[scale=0.4]{3_3(l)}
  \includegraphics[scale=0.4]{3_3(d)}
  
  \item С помощью алгоритма Томсона построим ДКА
  \item Q: \{1\} \{1,7\} \{9\} \{1,7,12\} \{9,14\} \{12\} \{14\} \{1,9,14\} \{1,14\} \{1,9\}
  \item \begin{tabular}{ | l | l | l | }
  \hline
    Узел & a & b \\ \hline
    1 & 1,7 & 9 \\
    1,7 & 1,7,12 & 9,14 \\
    9 & 12 & 14 \\
    1,7,12 & 1,7,12 & 1,9,14 \\
    9,14 & 12 & 1,14 \\
    12 & \emptyset & 1 \\
    14 & \emptyset & 1 \\
    1,9,14 & 1,7,12 & 1,9,14 \\
    1,14 & 1 & 1,9 \\
    1,9 & 1,7,12 & 9,14 \\
  \hline
  \end{tabular}
  \item Минимизируем полученный ДКА
  \item\includegraphics[scale=0.4]{3_3(f)}
  \includegraphics[scale=0.4]{3_3(m)}
  \newpage
  
  \item [(b+c)((ab)∗c+(ba)∗)∗]
  \item Построим НКА с использованием $\lambda$-переходов, затем удалим их
  
  \item \includegraphics[scale=0.4]{3_4(l)}
  \includegraphics[scale=0.4]{3_4(d)}
  
  \item С помощью алгоритма Томсона построим ДКА
  \item Q: \{1\} \{3,14\} \{11\} \{17\} \{14\} \{3,12\} \{18\}
  \item \begin{tabular}{| l | l | l | l | }
  \hline
    Узел & a & b & c\\ \hline
    1 & \emptyset & 3,14 & 3,14 \\
    3,14 & 11 & 17 & 14 \\
    11 & \emptyset & 3,12 & \emptyset \\
    17 & 18 & \emptyset & \emptyset \\
    14 & 11 & 17 & 14 \\
    3,12 & 11 & 17 & 3,14 \\
    18 & 11 & 17 & 14 \\
  \hline
  \end{tabular}
  \item Минимизируем полученный ДКА
  \item\includegraphics[scale=0.4]{3_4(f)}
  \includegraphics[scale=0.4]{3_4(m)}
  \newpage
  
  \item [(a + b)^{+}(aa + bb + abab + baba)(a + b)^{+}]
  \item Построим НКА с использованием $\lambda$-переходов, затем удалим их
  \item \includegraphics[scale=0.35]{3_5(l)}
  \includegraphics[scale=0.35]{3_5(d)}
  
  \item С помощью алгоритма Томсона построим ДКА
  \item Q: \{1\} \{3\} \{3,8\} \{3,15\} \{3,8,10\} \{3,12,15\} \{3,8,19\} \{3,15,17\} \{3,8,10,26\} \{3,12,15,26\} \{3,8,13,19\} \{3,12,15,20\} \{3,8,13,19,26\} \{3,15,17,26\} \{3,8,19,26\} \{3,12,14,15,20\} \{3,8,13,19,21\} \{3,12,14,15,20,26\} \{3,8,13,19,21,26\} \{3,12,15,20,26\}
  \item \begin{tabular}{| l | l | l | l | }
  \hline
    Узел & a & b \\ \hline
    1 & 2 & 2 \\
    2 & 3 & 4 \\
    3 & 5 & 6 \\
    4 & 7 & 8 \\
    5 & 9 & 10 \\
    6 & 11 & 8 \\
    7 & 5 & 12 \\
    8 & 15 & 14 \\
    9 & 9 & 10 \\
    10 & 13 & 14 \\
    11 & 5 & 16 \\
    12 & 17 & 8 \\
    13 & 9 & 18 \\
    14 & 15 & 14 \\
    15 & 9 & 20 \\
    16 & 19 & 14 \\
    17 & 9 & 18 \\
    18 & 19 & 14 \\
    19 & 9 & 18 \\
    20 & 19 & 14 \\
  \hline
  \end{tabular}
  \item Минимизируем полученный ДКА
  \item \includegraphics[scale=0.4]{3_5(f)}
  \includegraphics[scale=0.4]{3_5(m)}
  \newpage
\end{description}

%4 задание
\section{Определить является ли язык регулярным или нет}
\begin{description}
  \item [$L = \{(aab)^{n}b(aba)^{m}| n \geq 0, m \geq 0\}$]
  \item Язык является регулярным, так как возможно построить автомат
  \item \includegraphics[scale=0.4]{4_1}
  
  \item [$L = \{ uaav |u \in \{a, b\}^{*}, v \in a,b^{*}, |u|_b \geq |v|_a \}$]
  \item Для удобства, докажем что дополнение к языку является регулярным, воспользовавшись леммой о накачке
  \item $\overline{L} = \{ uaav |u \in \{a, b\}^{*}, v \in a,b^{*}, |u|_b > |v|_a \}$
    \begin{itemize}
      \item Зафиксируем n
      \item Возьмем слово $b^n aa a^{n+1} \in \overline{L}$
      \item Разобьем его на $x,y,z$ такие, что $|xy| \leq n$ и $|y| \geq 1$
      \begin{itemize}
        \item при 0 < m < n
        \item $x=b^{n-m}$
        \item $y=b^m$
        \item $z=aa a^{n+1}$
      \end{itemize}
      \item Тогда при накачке $y$ полученное слово $\notin \overline{L}$ 
    \end{itemize}
  \item Так как дополнение - нерегулярный язык, то и L нерегулярный.
  
  \item [$L = \{a^mw | w \in \{a, b\}^*, 1 \leq |w|_b \leq m\}$]
  \item Докажем, что язык регулярный с помощью дополнения
  \item $\overline{L} = \{a^mw | w \in \{a, b\}^*,1>|w|_b>m\}$
  \item Таким образом, так как не может быть отрицательного кол-ва букв, получаем язык $w=a^k, где k>0$
  \item Данный автомат легко построить
  \item \includegraphics[scale=0.4]{4_3}
  
  \item [$L = \{a^kb^ma^n | k = n \lor m > 0\}$]
  
  \item [$L = \{ucv | u \in \{a, b\}^*, v \in \{a, b\}^*, u \neq v^R\}$]
  \item Для удобства, докажем что дополнение к языку является регулярным, воспользовавшись леммой о накачке
  \item $\overline{L} = \{ucv | u \in \{a, b\}^*, v \in \{a, b\}^*, u = v^R\}$
    \begin{itemize}
      \item Зафиксируем n
      \item Возьмем слово $a^{2(n+1)} c a^{n+1} \in \overline{L}$
      \item Разобьем его на $x,y,z$ такие, что $|xy| \leq n$ и $|y| \geq 1$
      \begin{itemize}
        \item $x=a^{n-p}$
        \item $y=a^p$
        \item $z=a^{n+2}ca^{n+1}$
      \end{itemize}
      \item Тогда при накачке $y$ полученное слово $\notin \overline{L}$ 
    \end{itemize}
  \item Так как дополнение - нерегулярный язык, то и L нерегулярный.

\end{description}
\end{document}
